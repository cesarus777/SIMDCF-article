\section{Оптимизация}
\label{sec:Optimization}

Оптимизацией должен являться LLVM Pass из LLVM IR в LLVM IR в векторном
backend'е графического компилятора Intel. Для упрощения проектирования,
реализации и поддержки, оптимизация разделена на две части: анализ и собственно
трансформацию.

Перед рассмотрением анализа и трансформации,
проведем обзор возможных конструкций, которые ожидаются на вход оптимизации.

Так как рассматривается только структурированный поток управления, то можно
рассматривать только два атомарных случая: условный переход if-else и цикл.
На рисунках ~\ref{fig:if-else-simdcf-simple} и ~\ref{fig:loop-simdcf-simple}
представлен граф потока управления.
\begin{figure}
  \centering
  \includegraphics[scale=0.27]{Images/if-else-FE-colored.png}
  \caption{Схема потока управления для случая if-else}
  \label{fig:if-else-simdcf-simple}
\end{figure}
\begin{figure}
  \centering
  \includegraphics[scale=0.27]{Images/do-while-FE-colored.png}
  \caption{Схема потока управления для случая цикла}
  \label{fig:loop-simdcf-simple}
\end{figure}

Теперь же рассмотрим случай с вложенными базовыми конструкциями потока
управления: if-else вложенный в if-else (рисунок ~\ref{fig:nested-if-simdcf}) и
цикл с вложенным if-else (рисунок ~\ref{fig:loop-nested-if-simdcf}).
\begin{figure}
  \centering
  \includegraphics[scale=0.27]{Images/nested-if-FE-colored.png}
  \caption{Схема потока управления для случая if-else вложенного в if-else}
  \label{fig:nested-if-simdcf}
\end{figure}
\begin{figure}
  \centering
  \includegraphics[scale=0.27]{Images/do-while-nested-if-FE-colored.png}
  \caption{Схема потока управления для случая if-else вложенного в цикл}
  \label{fig:loop-nested-if-simdcf}
\end{figure}

Нетрудно заметить, что можно выделить регионы, содержащий в себе одну
конструкцию векторного потока управления определенного уровня вложенности.
Поэтому определим SIMD CF регион как регион, содержащий в себе одну конструкцию
предложенного ранее интерфейса для самого внешнего уровня вложенности,
доступного для данного региона. Как можно будет увидеть дальше, такой подход
позволит упростить анализ, позволяя сначала обнаруживать самые внешние SIMD CF
регионы, а после - вложенные. Тогда обобщенные SIMD CF регионы будут выглядеть
как на рисунках ~\ref{fig:generalized-if-simdcf} и ~\ref{fig:generalized-loop-simdcf}.
\begin{figure}
  \centering
  \includegraphics[scale=0.27]{Images/if-else-FE-generalized-colored.png}
  \caption{Схема if-else SIMD CF региона}
  \label{fig:generalized-if-simdcf}
\end{figure}
\begin{figure}
  \centering
  \includegraphics[scale=0.27]{Images/do-while-FE-generalized-colored.png}
  \caption{Схема if-else SIMD CF региона}
  \label{fig:generalized-loop-simdcf}
\end{figure}

После определения SIMD CF регионов можно свести задачу поиска заранее
оговоренных конструкций к поиску SIMD CF регионов самого внешнего уровня
вложенности, после чего искать вложенные SIMD CF регионы.

Шаг 1. Поиск условного перехода, похожего на векторный поток управления. Для
каждого базового блока проверяется его терминатор. Если это инструкция условного
перехода, то проверяется его условие, в противном случае конструкция не является
SIMD CF регионом. Если условием является результат вызова одного из обозначенных
ранее интринсиков, то идет переход к шагу 2.

Шаг 2. Проверяется структура потока управления и происходит попытка сопоставить
его либо с SIMD CF if/else, либо с SIMD CF циклом. Если сопоставление с одним из
заданных паттернов невозможно, то конструкция не является SIMD CF регионом. В
противном случае происходит переход к шагу 3.

Шаг 3. Проверка маскирования побочных эффектов. Для каждой инструкции, у которой
есть побочные эффекты проводится проверка, является ли такая инструкция
маскирована и если она маскирована, то проверяется, совпадает ли маска для
данной инструкции с условием перехода в эту дугу. Для вложенных регионов
проверяется, является ли маска данного региона подмножеством маски внешнего
региона. Если проверка неудачная - данный регион не является SIMD CF регионом.

Дополнение к шагу 3 для цикла. Проверяются фи-узлы для индуктивностей и пересчет
маски для каждого цикла. Если проверка неудачная - данный регион не является
SIMD CF регионом. В противном случае идет переход к шагу 4.

Дополнение к шагу 3 для if-else. Если кроме if также имеется else, то происходит
проверка, являются ли маски if и else строго противоположны друг другу. Если
проверка неудачная - данный регион не является SIMD CF регионом. В противном
случае идет переход к шагу 4.

Шаг 4. Данный регион является SIMD CF регионом. Аналогично происходит поиск
вложенных SIMD CF регионов для данного региона.

После сбора всей информации о SIMD CF регионах начинается их оптимизирующая
трансформация.

Для этого сначала рассмотрим текущую модель \texttt{goto}-\texttt{join} в
векторном backend'е графического компилятора Intel. Текущая модель показана на
схемах на рисунках ~\ref{fig:if-goto-join-BE} и ~\ref{fig:loop-goto-join-BE}.
\begin{figure}
  \centering
  \includegraphics[scale=0.27]{Images/if-else-BE-current.png}
  \caption{\texttt{goto}-\texttt{join} cхема if-else в backend'е}
  \label{fig:if-goto-join-BE}
\end{figure}
\begin{figure}
  \centering
  \includegraphics[scale=0.27]{Images/do-while-BE-current.png}
  \caption{\texttt{goto}-\texttt{join} cхема цикла в backend'е}
  \label{fig:loop-goto-join-BE}
\end{figure}

Попробуем трансформировать разработанный интерфейс в уже имеющуюся модель
\texttt{goto}-\texttt{join}. Для этого надо добавить \texttt{goto} для
вычисления условия условного перехода, вставить соответствующий \texttt{join},
после чего убрать маскирование побочных эффектов.

Тогда мы получим следующие схемы для трансформации представленные на рисунках
~\ref{fig:if-transform} и ~\ref{fig:loop-transform}.
\begin{figure}
  \centering
  \includegraphics[width=0.5\textwidth]{Images/if-else-BE.png}
  \caption{Трансформация if-else}
  \label{fig:if-transform}
\end{figure}
\begin{figure}
  \centering
  \includegraphics[width=0.5\textwidth]{Images/do-while-BE.png}
  \caption{Трансформация цикла}
  \label{fig:loop-transform}
\end{figure}
